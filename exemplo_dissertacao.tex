\documentclass[dissertacao]{iftex2024}

\addbibresource{referencias.bib}

\titulo{Modelo de Dissertação}
\autor{Marcos Roberto Ribeiro}
\local{Bambuí -- MG}
\data{2024-04-18}
\campus{\textit{Campus} Bambuí}
\curso{Mestrado}{Sustentabilidade e Tecnologia Ambiental}
\titulacao{Mestre}

\orientador{Nome da Orientador}
\linhapesquisa{Ecologia Aplicada}
\areaconcentracao{Educação e Sustentabilidade}
\membrobanca{Fulano de Tal}{Instituição do Fulano de Tal}
\membrobanca{Ciclano de Tal}{Instituição do Ciclano de Tal}
\fichacatalografica{ficha.pdf}

\resumo{Este trabalho é um modelo em {\latex} utilizando a classe \iftex.
Tal classe foi desenvolvida com base no manual de normalização de trabalhos acadêmicos do IFMG e nas normas relacionadas da Associação Brasileira de Normas Técnicas.
Este modelo apresenta uma estrutura básica com exemplos de elementos pré e pós-textuais.
Maiores informações sobre como utilizar a classe podem ser encontradas no manual da classe \iftex.}
\palavraschave{\iftex. Modelo. IFMG. \latex.}

\abstract{This work is a template in {\LaTeX} using the \iftex class.
This class was developed based on the academic work standardization manual of IFMG and the related norms of the Brazilian Association of Technical Standards.
This template presents a basic structure with examples of pre and post-textual elements.
Further information on how to use the class can be found in the \iftex class manual.}
\keywords{\iftex. Template. IFMG. \latex.}

\begin{document}

\maketitle

\chapter{INTRODUÇÃO}

A introdução desempenha um papel fundamental na preparação do leitor para o conteúdo que será abordado.
Ela começa contextualizando o tema, fornecendo informações relevantes sobre o assunto, sua importância e seu contexto mais amplo na área de estudo.
Além disso, a introdução deve fornecer justificativas convincentes para a realização da pesquisa, identificando lacunas no conhecimento existente, relevância prática ou teórica do tema e importância potencial dos resultados.

Destaca-se que as contribuições esperadas do trabalho para a área de estudo, que podem incluir avanços teóricos e práticos, implicações políticas ou sociais, entre outros.
Por fim, a introdução é geralmente concluída com um parágrafo que resume brevemente o objetivo geral do trabalho, reiterando os objetivos estabelecidos anteriormente.
É essencial que essa seção seja redigida com clareza e coesão para capturar a atenção do leitor e estabelecer uma base sólida para o restante do trabalho.

Por fim, é importante observar o regulamento e as normas de formatação e de elaboração de trabalhos de conclusão de curso do IFMG \cite{ifmg:2020:manual,ifmg:2021:tcc}.
Além disso, é interessante consultar o manual da classe \iftex para conhecer mais sobre as configurações e exemplos de uso \cite{ribeiro:2024:iftex}.

\section{Objetivos}

Os objetivos definem claramente o propósito e as metas do trabalho, devem ser específicos, mensuráveis, alcançáveis, relevantes e limitados no tempo.
Os objetivos podem ser divididos em objetivo geral e objetivos específicos.

\subsection{Objetivos geral}

O objetivo geral é a meta principal do trabalho, definindo o propósito geral do estudo.
O objetivo geral deste trabalho é apresentar um modelo de documento usando a classe \iftex.

\subsection{Objetivos específicos}

Os objetivos específicos são metas detalhadas que precisam ser alcançadas para atingir o objetivo geral\footnote{Recomenda-se que não seja estabelecida uma quantidade muito grande de objetivos específicos.}.
Eles direcionam as ações do trabalho e fornecem uma estrutura clara para o trabalho.
Como exemplo podemos estabelecer os seguintes objetivos específicos:
\begin{enumerate}
 \item Apresentar exemplos de elementos pré-textuais;
 \item Mostrar uma estrutura básica de documento;
 \item Exemplificar o uso de elementos pós-textuais.
\end{enumerate}

\chapter{FUNDAMENTOS TEÓRICOS}

A seção de fundamentos teóricos fornece uma base teórica sólida para o estudo, contextualizando o trabalho dentro do corpo existente de conhecimento na área.
A seção de fundamentos teóricos fornece a base conceitual e contextual para o seu estudo.
É importante escrevê-la de forma clara, organizada e fundamentada em pesquisas anteriores, destacando a relevância e originalidade do seu trabalho.

\chapter{METODOLOGIA}

A metodologia descreve os métodos e procedimentos utilizados na pesquisa.
Ela inclui detalhes sobre o design do estudo, a coleta e análise de dados, além da justificativa das escolhas metodológicas.
É essencial para garantir a validade e confiabilidade dos resultados.
A metodologia deve ser clara e detalhada o suficiente para que outros pesquisadores possam replicar o estudo.

\chapter{DESENVOLVIMENTO}

A seção de desenvolvimento apresenta e discute os resultados do trabalho.
Inicialmente, os resultados são apresentados de forma objetiva, seguidos por uma discussão que os relaciona aos objetivos e à revisão de literatura.
A interpretação dos resultados à luz das teorias é essencial, assim como a comparação com estudos anteriores.
Finalmente, é importante reconhecer as limitações do estudo e sugerir direções futuras.
Essa seção contribui para a compreensão do tema e o avanço do conhecimento na área.

\chapter{CONCLUSÃO}

A conclusão resume os principais pontos discutidos e apresenta as conclusões alcançadas a partir do trabalho.
Ela destaca as descobertas mais significativas, sua relação com a literatura existente e suas implicações práticas ou teóricas.
Além disso, a conclusão reafirma os objetivos do trabalho e sugere áreas para futuras investigações.
É importante evitar a introdução de novas informações e manter a conclusão concisa e alinhada com os objetivos e resultados do estudo.

\chapter*{REFERÊNCIAS}

\printbibliography

\end{document}
